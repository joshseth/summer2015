\documentclass[a4paper, 11 pt]{article}
\usepackage{amsmath, amssymb, color, setspace, graphicx, dsfont, pdfpages, float, wrapfig, indentfirst}
\usepackage[left=0.5in, right=0.5in]{geometry}
\usepackage[font=small,labelfont=bf]{caption}
\begin{document}

\begin{enumerate}
  \item Pleiotropy

  \item ``Discreteness" of fitness
    Use $\mathbb{R}$ instead of $\mathbb{Z}$; take a function of the Hamming distances between TFs and TFBS.
  \item Snowball Mechanism
    Incompatibility vs. Time is not an adequate measure of the Snowball effect. We are really interested in whether or not the number of possible DMI mutations go up quadratically with the number of changes. Therefore, we should look at the number of fixed mutants that cause death when placed in an ancestor as a function of the number of fixed mutations. This relationship should be exponential.  
  \item Disentagle the influence of sequence drift vs.\ rewiring underlying Incompatibility
  \item Allow generation overlap
  \item Predict \emph{in vivo} DSD rates. 
  \item DSD rate estimate and a critical migration value that entails speciation.
  \item Check for bugs in the code including possible memory leaks. 
    Why is the simulation terminated after approximatelt 68,000 generations?
  \item Consider extending the regulatory region length. Would we see more mutations of small effect?
  \item Quantify the degree of constraint. (1) nucleodtide: the number of substitutions vs. the mutation rate (\emph{i.e.} expected without selection). (2) network(G): Euclidean distance vs. expected under just mutation. 
  \item Is modularity selected for? 
    See if ``modularity" goes up as a population size increases or down as selection decreases. 
\end{enumerate}



\end{document}

